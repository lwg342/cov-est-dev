This paper considers the problem of incorporating auxialiary data such as textual-analysis data into the estimation of large covariance matrices. It can be shown that by incorporating information about locations of important links we can relax the sparsity conditions of the thresholding estimators and in simulations we show the proposed Network Guided Estimator has superior performance in finite samples. We have also applied the Network Guided Estimator in the construction of minimum-variance portfolio in a preliminary empirical application. There are several improvements that we are undertaking.

Firstly, the construction of good estimator \(\hat{L}_{ij}\) for the important locations is an important question. It's apparent from the simulations that the quality of \(\hat{L}_{ij}\) will affect the estimation error. We have used a straightforward estimator in the empirical study, but it's not completely satisfactory and we need a more systematic way of constructing the \(\hat{L}\). Secondly, we are expanding the set of auxialiary networks beyond the RavenPack news data to include Hoberg's similarity score and IBES co-covarage data, as well as applying the technique on a larger dataset. 

% Firstly, we are applying the covaraince estimation technique on portfolio construction, following the problem considered in \cite{ledoit2004HoneyShrunk} and \cite{ledoit2017NonlinearShrinkage}, where the estimation of the sparse covaraince matrices are vital for constructing the minimum-variance portfolio. 

% Secondly, the method can be applied to study spatial-APT under large \(N\) case. \cite{kou2018AssetPricing} find that common risk factors are insufficient to capture all the significant inter-dependencies in asset returns, and local interactions are also important.  Spatial-APT and spatial CAPM type of models have not been popular in large N case since the measure of contiguity is challenging. Our method can uncover contemporaneously correlated entities by combining market-based information and auxiliary network information, thus providing a natural contiguity measure. Relying solely on either statistical methods or external network information is not as desirable as the links identified by the former are hard to interpret and the external network may miss some important links.



Although here we use network information to the estimation of large static covariance matrix, similar ideas can be extended to the estimation of large dynamic covariance matrix. For example, dynamic network information could be incorporated into the conditioning information set like in \cite{chen2019new}.
